%%%%%%%%%%%%%%%%%%%%%%%%%%%%%%%%%%%%%%%%%%%%%%%%%%%%%%%%%%%%%%%%%%%%%%%%%%%%%%%%
%2345678901234567890123456789012345678901234567890123456789012345678901234567890
%        1         2         3         4         5         6         7         8

\documentclass[letterpaper, 12 pt, conference]{ieeeconf}  % Comment this line out
                                                          % if you need a4paper
%\documentclass[a4paper, 10pt, conference]{ieeeconf}      % Use this line for a4
                                                          % paper

\IEEEoverridecommandlockouts                              % This command is only
                                                          % needed if you want to
                                                          % use the \thanks command
\overrideIEEEmargins
% See the \addtolength command later in the file to balance the column lengths
% on the last page of the document



% The following packages can be found on http:\\www.ctan.org
\usepackage{graphicx} % for pdf, bitmapped graphics files
\usepackage{bm}
\newcommand{\uvec}[1]{\boldsymbol{\hat{\textbf{#1}}}}
%\usepackage{epsfig} % for postscript graphics files
%\usepackage{mathptmx} % assumes new font selection scheme installed
%\usepackage{times} % assumes new font selection scheme installed
%\usepackage{amsmath} % assumes amsmath package installed
%\usepackage{amssymb}  % assumes amsmath package installed

\title{\LARGE \bf
Predicting Online Shoppers’ Purchasing Intention
}

\author{
Anthony Stefanuto$^{1}$, Luke Jang$^{2}$, Aimee Tang$^{3}$, and Jacklyne XXXXXX$^{4}$%
\thanks{}%
\thanks{$^{1}$Anthony Stefanuto, Computer Science, Western University.}%
\thanks{$^{2}$Luke Jang, Computer Science, Western University.}%
\thanks{$^{3}$Aimee Tang, Statistics, Western University.}%
\thanks{$^{4}$Jacklyne XXXXXX, Statistics, Western University.}%
}


\begin{document}

\maketitle
\thispagestyle{empty}
\pagestyle{empty}

%%%%%%%%%%%%%%%%%%%%%%%%%%%%%%%%%%%%%%%%%%%%%%%%%%%%%%%%%%%%%%%%%%%%%%%%%%%%%%%%
\begin{abstract} 
How to conceal objects from electromagnetic radiation has been a hot research topic. Radar is an object detection system that uses Radio waves to determine the range , angle, or velocity. A radar transmit radio waves or microwaves that reflect from any object in their path. A receive radar is typically the same system as transmit radar, receives and processes these reflected wave to determine properties of object. Different organizations are working onto hide object from the radar in outer space. Any confidential object can be taken through space without being detected by the enemies. This calls for necessity of devising new method to conceal an object electromagnetically. 
\end{abstract}

%%%%%%%%%%%%%%%%%%%%%%%%%%%%%%%%%%%%%%%%%%%%%%%%%%%%%%%%%%%%%%%%%%%%%%%%%%%%%%%%
\section{Introduction}

The growth of e-commerce platforms depends heavily on how businesses interact with consumers. These interactions create copious amounts of behavioural and transactional data. Comprehension of this information creates opportunities to improve market strategies, increasing conversion rates. A predictive analysis can help businesses allocate time and resources more efficiently. Targeting customers with a high purchase intention increases profits while reducing wasted advertising spend.
The task of predicting customer purchase intention must be navigated cautiously. User behaviour on websites is highly variable and influenced by numerous factors, including browsing duration, the types of pages visited, and bounce rate. Conventional rules and analysis fail to capture advanced interactions and data correlations. This is where the implementation of machine learning (ML) models can identify patterns in data, facilitating data-driven business decisions.
This study uses the Online Shoppers Purchasing Intention Dataset from the UCI Machine Learning Repository to classify user sessions as either purchase or non-purchase. Using this dataset, we investigate how user engagement metrics correlate with conversion rates. This study makes three main contributions:
We conduct a comprehensive exploratory data analysis (EDA) to understand trends, correlations, and target variable relationships


We implement and compare three machine learning models, including Logistic Regression, Decision Tree, and Random Forest, to evaluate their predictive performance.


We comprehend the key features driving purchase decisions, providing valuable insights for e-commerce platforms.
The remainder of this report is organized as follows. Section II discusses related research on purchase intention and online behavioural analytics. Section III details the dataset, preprocessing, and model design. Section IV presents and discusses experimental results, and Section V concludes with implications and potential directions for future work.

%%%%%%%%%%%%%%%%%%%%%%%%%%%%%%%%%%%%%%%%%%%%%%%%%%%%%%%%%%%%%%%%%%%%%%%%%%%%%%%%
\section{Related Work}

xxxxxxxxxxxxxxxxxxxxxxxxxxxxxxxxxx

%%%%%%%%%%%%%%%%%%%%%%%%%%%%%%%%%%%%%%%%%%%%%%%%%%%%%%%%%%%%%%%%%%%%%%%%%%%%%%%%
\section{Methodology / Proposed Method}

\subsection{Dataset Description}
xxxxxxxxxxxxxxxxxxxxxxxxxxxxxxxxx

\subsection{Data Preprocessing}
xxxxxxxxxxxxxxxxxxxxxxxxxxxxxx

\subsection{Algorithm Design}
xxxxxxxxxxxxxxxxxxxxxxxx

\subsection{Model Training}
xxxxxxxxxxxxxxxxxxxxxxxxxxxxx

%%%%%%%%%%%%%%%%%%%%%%%%%%%%%%%%%%%%%%%%%%%%%%%%%%%%%%%%%%%%%%%%%%%%%%%%%%%%%%%%
\section{Results and Discussion}

xxxxxxxxxxxxxxxxxxxxxxx

%%%%%%%%%%%%%%%%%%%%%%%%%%%%%%%%%%%%%%%%%%%%%%%%%%%%%%%%%%%%%%%%%%%%%%%%%%%%%%%%
\section{Conclusion}

xxxxxxxxxxxxxxxxxxxxxxxxx

%%%%%%%%%%%%%%%%%%%%%%%%%%%%%%%%%%%%%%%%%%%%%%%%%%%%%%%%%%%%%%%%%%%%%%%%%%%%%%%%
\begin{thebibliography}{99}

\bibitem{c1} Moro, S., Rita, P., and Cortez, P. (2016). "A Data-Driven Approach to Predict the Success of Bank Telemarketing." \textit{Decision Support Systems}, 62, 22–31.
\bibitem{c2} UCI Machine Learning Repository (2018). "Online Shoppers Purchasing Intention Dataset."
\bibitem{c3} Pedregosa, F. et al. (2011). "Scikit-learn: Machine Learning in Python." \textit{Journal of Machine Learning Research}, 12, 2825–2830.
\bibitem{c4} Powers, D. M. W. (2011). "Evaluation: From Precision, Recall and F-measure to ROC, Informedness, Markedness and Correlation." \textit{Journal of Machine Learning Technologies}, 2(1), 37–63.

\end{thebibliography}

\end{document}
